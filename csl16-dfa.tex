% arara: pdflatex
% arara: bibtex
% arara: pdflatex
% arara: pdflatex
\documentclass[a4paper,USenglish]{lipics-v2016}
%for section-numbered lemmas etc., use "numberwithinsect"
 
\usepackage{microtype}%if unwanted, comment out or use option "draft"

%\graphicspath{{./graphics/}}%helpful if your graphic files are in another directory

\bibliographystyle{plainurl}% the recommended bibstyle


\usepackage{csl-dfa}% custom macros


% Author macros::begin %%%%%%%%%%%%%%%%%%%%%%%%%%%%%%%%%%%%%%%%%%%%%%%%
\title{A sample title\footnote{This work was partially supported by someone.}}
\titlerunning{A Sample Running Title}% optional

\author{Henry DeYoung}
\author{Frank Pfenning}
\affil{Computer Science Department, Carnegie Mellon University, Pittsburgh, PA USA\\
  \texttt{\{hdeyoung,fp\}@cs.cmu.edu}}
\authorrunning{H. DeYoung and F. Pfenning}

\Copyright{Henry DeYoung and Frank Pfenning}%LIPIcs license is "CC-BY";  http://creativecommons.org/licenses/by/3.0/

\subjclass{Dummy classification -- please refer to \url{http://www.acm.org/about/class/ccs98-html}}% mandatory: Please choose ACM 1998 classifications from http://www.acm.org/about/class/ccs98-html . E.g., cite as "F.1.1 Models of Computation". 
\keywords{Dummy keyword -- please provide 1--5 keywords}% mandatory: Please provide 1-5 keywords
% Author macros::end %%%%%%%%%%%%%%%%%%%%%%%%%%%%%%%%%%%%%%%%%%%%%%%%%

%Editor-only macros:: begin (do not touch as author)%%%%%%%%%%%%%%%%%%%%%%%%%%%%%%%%%%
\EventEditors{John Q. Open and Joan R. Acces}
\EventNoEds{2}
\EventLongTitle{42nd Conference on Very Important Topics (CVIT 2016)}
\EventShortTitle{CVIT 2016}
\EventAcronym{CVIT}
\EventYear{2016}
\EventDate{December 24--27, 2016}
\EventLocation{Little Whinging, United Kingdom}
\EventLogo{}
\SeriesVolume{42}
\ArticleNo{23}
% Editor-only macros::end %%%%%%%%%%%%%%%%%%%%%%%%%%%%%%%%%%%%%%%%%%%%%%%

\begin{document}

\maketitle

\begin{abstract}
Lorem ipsum dolor sit amet, consectetur adipiscing elit. Praesent convallis orci arcu, eu mollis dolor. Aliquam eleifend suscipit lacinia. Maecenas quam mi, porta ut lacinia sed, convallis ac dui. Lorem ipsum dolor sit amet, consectetur adipiscing elit. Suspendisse potenti. 
 \end{abstract}

\section{Introduction}\label{sec:introduction}


\begin{lstlisting}[caption={Useless code},label=list:8-6,captionpos=t,float,abovecaptionskip=-\medskipamount]
for i:=maxint to 0 do 
begin 
    j:=square(root(i));
end;
\end{lstlisting}


\section{Intuitionistic linear logic with least fixed points}\label{sec:linear-logic}

In this \lcnamecref{sec:linear-logic}, we describe a fragment of intuitionistic linear logic with least fixed points.
Propositions are given by the grammar
\begin{equation*}
  A \Coloneqq \plus*[\ell \in L]{\ell{:}A_{\ell}} \mid A_1 \tensor A_2 \mid \one \mid \indtype{X.A} \mid X
  \,,
\end{equation*}
where $\plus*[\ell \in L]{\ell{:}A_{\ell}}$ is a labeled $n$-ary additive disjunction over the set $L$ of labels, $A_1 \tensor A_2$ and $\one$ are multiplicative conjunction and its unit, and $\indtype{X.A}$ is the least fixed point that is the least solution to the equation $X = A$.
Because all of the connectives in this fragment are covariant, notice that the least fixed points necessarily satisfy the standard strict positivity condition that guarantees well-definedness.
% We enforce the standard strict positivity requirement on inductive types to ensure that they are well-defined.
%
As usual, contexts, $\lctx$, are finite multisets of propositions:
\begin{equation*}
  \lctx \Coloneqq \lctxe \mid \lctx, A
  \,.
\end{equation*}

We use a sequent calculus presentation in which sequents have the form $\seq{\lctx |- A}$.
The right and left introduction rules for $\plus$, $\tensor$, and $\one$, along with cut and identity rules are given in \cref{fig:linear-logic}.
%
\begin{figure}
  \begin{mathpar}
    \infer[{\jrule[_A]{CUT}}]{\seq{\lctx , \lctx' |- \spawn{P; Q} :: z:C}}{
      \seq{\lctx |- P :: x:A} &
      \seq{\lctx' , x:A |- Q :: z:C}}
    \qquad
    \infer[{\jrule[_A]{ID}}]{\seq{x:A |- \fwd{z}{x} :: z:A}}{
      }
    %
    \\
    %
    \infer[\rrule{\plus}]{\seq{\lctx |- \select{x}{\kay; P} :: x:\plus*[\ell \in L]{\ell{:}A_{\ell}}}}{
      \seq{\lctx |- P :: x:A_{\kay}} &
      \text{($\kay \in L$)}}
    \qquad
    \infer[\lrule{\plus}]{\seq{\lctx' , x:\plus*[\ell \in L]{\ell{:}A_{\ell}} |- \case{x}[\ell \in L]{\ell => Q_{\ell}} :: z:C}}{
      \forallseq{\ell \in L}{\seq{\lctx' , x:A_{\ell} |- Q_{\ell} :: z:C}}}
    %
    \\
    %
    \infer[\rrule{\tensor}]{\seq{\lctx_1 , \lctx_2 |- \send{x}{P_1; P_2} :: x:A_1 \tensor A_2}}{
      \seq{\lctx_1 |- P_1 :: y:A_1} &
      \seq{\lctx_2 |- P_2 :: x:A_2}}
    \qquad
    \infer[\lrule{\tensor}]{\seq{\lctx' , x:A_1 \tensor A_2 |- \recv{x}{y <- Q} :: z:C}}{
      \seq{\lctx' , y:A_1 , x:A_2 |- Q :: z:C}}
    %
    \\
    %
    \infer[\rrule{\one}]{\seq{\lctxe |- \close{x} :: x:\one}}{
      }
    \qquad
    \infer[\lrule{\one}]{\seq{\lctx' , x:\one |- \wait{x}{Q} :: z:C}}{
      \seq{\lctx' |- Q :: z:C}}
  \end{mathpar}
  \caption{Inference rules for a sequent calculus presentation of a fragment of linear logic\label{fig:linear-logic}}
\end{figure}
%
Rather than giving right and left introduction rules for $\mu$, it will be convenient to treat inductive types and their proofs \vocab{equi-recursively}.
We will identify $\indtype{X.A}$ with its unrolling $[(\indtype{X.A})/X]A$, 


\section{Proofs as \aclp*{DFA}}\label{sec:proofs-as-dfas}

\subsection{\Aclp*{DFA}}\label{sec:dfas}

A \vocab{deterministic finite automaton}~(\acs{DFA}) $M$ is a 5-tuple $(Q, \Sigma, \delta, q_0, F)$ where
$Q$ is a finite set of states,
$\Sigma$ is a finite alphabet of input symbols $a$,
$\delta\colon Q \times \Sigma \to Q$ is a total transition function,
% function that describes the state transitions upon reading an input letter,
$q_0 \in Q$ is the initial state, and
$F \subseteq Q$ is a set of accepting states.
% For convenience in speaking of \emph{the} language accepted by the \ac{DFA} $M$, it is typical to also fix a starting state, $q_0 \in Q$, for all computations of $M$. % the \ac{DFA} $M$.
% We choose not to do so.
The set of all finite strings over alphabet $\Sigma$ is written as $\Sigma^*$, with $\emp$ denoting the empty string.

A \vocab{computation} of % the automaton
$M$ on the input string $w = a_1 a_2 \dotsb a_n \in \Sigma^*$
is a sequence of states $q_0, q_1, \dots, q_n$ such that $\delta(q_i, a_{i+1}) = q_{i+1}$ for each $0 \leq i < n$.
% The \ac{DFA}
$M$ is said to \vocab{accept} string $w$ if there exists a computation on $w$ that ends in an accepting state, $q_n \in F$; otherwise, $M$ is said to \vocab{reject} string $w$.

Alternatively, one may extend the transition function $\delta$, which operates on symbols, to a function $\delta^*\colon Q \times \Sigma^* \to Q$ that operates on strings.
% the lifting of the transition function $\delta$ from letters to words, by induction on the length of the input word:
\begin{align*}
  \delta^*(q, \emp) &= q \\
  \delta^*(q, a w') &= \delta^*(q', w') \text{\,, where $\delta(q, a) = q'$}
\end{align*}
It's straightforward to show, by induction on the length of the string $w$, that $M$ accepts $w$ if and only if $\delta^*(q_0, w) \in F$.

Finally, the \vocab{language recognized} by $M$ is the set of all strings that are accepted by $M$.


\begin{figure}
  \begin{tikzpicture}
    \graph [grow right sep=2em, math nodes, nodes={state}] {
      0 / "\mathclap{\state{even}}" [initial, initial text=, accepting]
       -> [bend left, "$\isym{b}$"]
      1 / "\mathclap{\smash{\state{odd}}\vphantom{\state{even}}}"
       -> [bend left, "$\isym{b}$"]
      0;
      %
      0 -> [loop above, "$\isym{a}$"] 0;
      1 -> [loop above, "$\isym{a}$"] 1;
    };
  \end{tikzpicture}
  %
  % \begin{gather*}
  %   even = \caseL{
  %            \isym{a} => even
  %          | \isym{b} => odd
  %          | \eow => \waitL{\selectR{\acc ; \closeR}}}
  %   %
  %   \\
  %   %
  %   odd = \caseL{
  %           \isym{a} => odd
  %         | \isym{b} => even
  %         | \eow => \waitL{\selectR{\rej ; \closeR}}}
  % \end{gather*}
  \caption{The transition diagram for \iac*{DFA} that recognizes strings over the alphabet $\{\isym{a}, \isym{b}\}$ that contain an even number of $\isym{b}$s.\label{fig:dfa-example}}
\end{figure}
%
\begin{example}
\Cref{fig:dfa-example} shows the transition diagram for \iac{DFA} that recognizes finite strings over the alphabet $\Sigma = \{\isym{a}, \isym{b}\}$ that contain an even number of $\isym{b}$s.
The starting state, as indicated by the unlabeled arrow, is $\state{even}$; it is also an accepting state, as indicated by its doubled outline.
% corresponds to having read an even number of $\isym{b}$s so far; likewise, the state $\state{odd}$ corresponds to having read an odd number of $\isym{b}$s so far.
This \ac{DFA} accepts the string $\isym{b} \isym{a} \isym{b}$ because there is a path corresponding to $\isym{b} \isym{a} \isym{b}$ from the starting state, $\state{even}$, to an accepting state; % , namely $\state{even}$.
this \ac{DFA} rejects the string $\isym{b} \isym{a}$ because the path corresponding to $\isym{b} \isym{a}$ from the starting state ends at a rejecting state.
\end{example}

Although some definitions of \acp{DFA} allow the transition function $\delta$ to be a partial function, notice that we demand that $\delta$ is total.
Totality ensures that \iac{DFA} never gets stuck while reading an input symbol;
this will be crucial for the correspondence between \ac{DFA} computations and cut reductions that we give in \cref{sec:curry-howard-dfas}.
In demanding totality, there is no loss of expressiveness: one can always introduce a distinguished rejecting state that acts as a sink for any transitions that would have been left undefined under a partial transition function.

% Notice our demand that the transition function $\delta$ be a total function.
% Some definitions of \acp{DFA} allow transition functions to be partial functions, but notice that we demand the transition function be a total function.
% In making this demand, there is no loss of expressiveness, however, because one may always introduce a sink state that is not accepting.


\subsection{An adequate encoding of finite strings}

% We will encode finite strings over the alphabet $\Sigma$ as proofs of 
% %
% Specifically, the encoding $\sp{}\colon \Sigma^* \to (\seq{\lctxe |- \Tape[\Sigma]})$ is defined inductively by
% \begin{align*}
%   \sp{\emp} &= \select{r}{\$; \close{r}} \\
%   \sp{aw} &= \select{r}{a; \sp{w}}
% \end{align*}

Strings $w \in \Sigma^*$ are finite lists --- lists of symbols drawn from the finite alphabet $\Sigma$.
In type theory, the type, $\List[\alpha]$, of finite lists with elements of type $\alpha$ is usually defined as the least solution of the type equation $X_\alpha = (\alpha \times X_\alpha) + \one$, or, when the type $\alpha$ is finitely inhabited, the isomorphic $X_\alpha = X_\alpha + \dots + X_\alpha + \one$.
Finite lists are then values built inductively from the constructor $\mathsf{nil} : \one \to \List[\alpha]$ and the family of constructors $\mathsf{cons}_a : \List[\alpha] \to \List[\alpha]$, one for each value $a : \alpha$.

By analogy, we will encode strings over $\Sigma$ as process terms of type $\seq{\lctxe |- \Tape[\Sigma]}$, where the type $\Tape[\Sigma]$ is the least solution of the equation
\begin{equation*}
  X = \plus*[a \in \Sigma \cup \{\eow\}]{a{:}A_a}
  \quad\text{where}\quad
  A_a = \begin{cases*}
          X & if $a \in \Sigma$ \\
          \one & if $a = \eow$
        \end{cases*}
\end{equation*}
In other words, $\Tape[\Sigma]$ is the inductive type $\indtype{X.\, \plus*[a \in \Sigma \cup \{\eow\}]{a{:}A_a}}$.
(From here on, we will indulge in a slight abuse of notation, writing $\plus*[a \in \Sigma]{a{:}A_a, \eow{:}A_{\eow}}$ instead of the more pedantic $\plus*[a \in \Sigma \cup \{\eow\}]{a{:}A_a}$.)

Specifically, we define the encoding $\strp{{-}}\colon \Sigma^* \to (\seq{\lctxe |- \Tape[\Sigma]})$ inductively by
\begin{align*}
  \strp{\emp} &= \selectR{\eow ; \closeR} \\
  \strp{a w'} &= \selectR{a ; \strp{w'}}
  \,.
\end{align*}
In this way, the process term $\selectR{\eow ; \closeR}$ is rather like $\mathsf{nil}$, marking the end of the list, and $\selectR{a ; {-}}$ is rather like $\mathsf{cons}_a$.
For example, with the alphabet $\{\isym{a} , \isym{b}\}$, the word $\isym{b} \isym{a}$ is represented by the proof $\strp{\isym{b} \isym{a}} = \selectR{\isym{b} ; \selectR{\isym{a} ; \selectR{\eow ; \closeR}}}$.

% In a similar way, we will represent words over $\Sigma$ as proofs of $\seq{\lctxe |- \Tape[\Sigma]}$ where $\Tape[\Sigma]$ is the least solution of the type equation $X = \plus*[a \in \Sigma \cup \{\eow\}]{a{:}A_a}$ with $A_a = X$ if $a \in \Sigma$ and $A_{\eow} = \one$.
% , namely the inductive type $\indtype{X.\, \plus*[a \in \Sigma]{a{:}X , \eow{:}\one}}$.
% Specifically, we represent word $w \in \Sigma^*$ as $\wdpf{w}$, where $\wdpf{{-}}$ is the least function satisfying
% \begin{align*}
%   \wdpf{(\emp)} &= \selectR{\eow ; \closeR} \\
%   \wdpf{(a w')} &= \selectR{a ; \wdpf{(w')}}
%   \,.
% \end{align*}
% For example, if $\Sigma = \{\isym{0} , \isym{1}\}$, then the word $\isym{1} \isym{0}$ is represented by $\wdpf{(\isym{1} \isym{0})} = \selectR{\isym{1} ; \selectR{\isym{0} ; \selectR{\eow ; \closeR}}}$


This representation of strings as processes of a particular type is adequate:
\begin{theorem}
  Strings over the alphabet $\Sigma$ are in bijective correspondence with cut-free processes of type $\seq{\lctxe |- \Tape[\Sigma]}$.
  % The representation function $\wdpf{{-}}$ is a bijection.
\end{theorem}
\begin{proof}
  Define a total function $\pfwd{{-}}$ from processes of type $\seq{\lctxe |- \Tape[\Sigma]}$ to strings over % the alphabet
  $\Sigma$:
  \begin{gather*}
    \pfwd{(\selectR{\eow ; \closeR})} = \emp \\
    \pfwd{(\selectR{a ; P'})} = a \pfwd{(P')} \,,\, \text{if $a \neq \eow$}
    \,.
  \end{gather*}
  To see that $\pfwd{{-}}$ is total, take an arbitrary % proof $P$ of $\seq{\lctxe |- \Tape[\Sigma]}$., that is, a 
  process term $P$ such that $\aseq{\lctxe |- P : \Tape[\Sigma]}$ and consider its
  % outermost constructor.
  possible forms.
  Unrolling the inductive type $\Tape[\Sigma]$, the process term $P$ is such that $\aseq{\lctxe |- P : \plus*[a \in \Sigma]{a{:}\Tape[\Sigma] , \eow{:}\one}}$.
  Only the $\rrule{\plus}$ rule could possibly construct such a $P$.
  Therefore, the proof term $P$ has one of two forms: either
  \begin{itemize}%[label=\itshape\arabic*)]
  \item $P = \selectR{a ; P'}$ for some $a \in \Sigma$ and some process $P'$ such that $\aseq{\lctxe |- P' : \Tape[\Sigma]}$; or
  \item $P = \selectR{\eow ; P'}$ for some process term $P'$ such that $\aseq{\lctxe |- P' : \one}$.
  \end{itemize}
  The second form can be refined to $P = \selectR{\eow ; \closeR}$ because only the $\rrule{\one}$ rule can construct the requisite $P'$.
  % a $P'$ such that $\aseq{\lctxe |- P' : \one}$.
  The function $\pfwd{{-}}$ is total because it is defined for these two possible forms of $P$.

  It's easy to show that $\pfwd{{-}}$ is both a left and a right inverse of $\wdpf{{-}}$ and, consequently, that $\wdpf{{-}}$ is a bijection between strings over $\Sigma$ and proofs of $\seq{\lctxe |- P : \Tape[\Sigma]}$.
\end{proof}

\subsection{Encoding \acs*{DFA} states as proofs}

As defined in \cref{sec:dfas}, the essence of \iac{DFA} is its transition function, $\delta\colon Q \times \Sigma \to Q$, which induces a function $\delta^*(q_0, {-})\colon \Sigma^* \to Q$ that can be used to characterize the language accepted by the \ac{DFA}.
The transition function 


\section{Subtyping}

\begin{equation*}
  \seq{\Tape[\Sigma] |- even : \Ans}
\end{equation*}

\begin{align*}
  \EvenB &= \plus*{a:\EvenB , b: \OddB , \eow: \one} \\
   \OddB &= \plus*{a: \OddB , b:\EvenB             }
\end{align*}


\begin{align*}
  \lnot (\plus*[a \in \Sigma]{a: A_a , \eow: \one})
    &= \plus*[a \in \Sigma]{a: \lnot A_a}
  \\
  \lnot (\plus*[a \in \Sigma]{a: A_a})
    &= \plus*[a \in \Sigma]{a: \lnot A_a , \eow: \one}
  \\
  \lnot (\indtype{X.A})
    &= \indtype{\overline{X}.\, \lnot A}
  \\
  \lnot X &= \overline{X}
\end{align*}


\section{Closure properties by cut elimination}

\begin{align*}
  \MoveEqLeft[1]
  \spawn{q; not} \\
    &= \spawn{(\caseL[a \in \Sigma]{a => q_a | \eow => \waitL{\selectR{F(q); \closeR}}});
              not} \\
    &\reduces \caseL[a \in \Sigma]{a => \spawn{q_a; not} | \eow => \spawn{(\waitL{\selectR{F(q); \closeR}}); not}} \\
    &\reduces \caseL[a \in \Sigma]{a => \spawn{q_a; not} | \eow => \waitL{(\spawn{(\selectR{F(q); \closeR}); not})}} \\
    &= \caseL[a \in \Sigma]{a => \spawn{q_a; not} | \eow => \waitL{(\spawn{(\selectR{F(q); \closeR}); (\caseL{\acc => \waitL{\selectR{\rej; \closeR}} | \rej => \waitL{\selectR{\acc; \closeR}}})})}} \\
    &\reduces \caseL[a \in \Sigma]{a => \spawn{q_a; not} | \eow => \waitL{(\spawn{(\selectR{F(q); \closeR}); (\caseL{\acc => \waitL{\selectR{\rej; \closeR}} | \rej => \waitL{\selectR{\acc; \closeR}}})})}}
\end{align*}


\section{Subsequential transducers}



\section{Conclusion}\label{sec:conclusion}

\subparagraph*{Acknowledgements.}

We want to thank \dots


%%
%% Bibliography
%%

\bibliography{csl2016}

\end{document}
